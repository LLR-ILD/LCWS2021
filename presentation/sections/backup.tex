\newcounter{finalframe}
\setcounter{finalframe}{\value{framenumber}}

\xsection*{myblue}{Back-up}[extras/pr_ilc_cavity_resized]

\begin{frame}{Expected counts per (category, BR) pair}
  \label{overlay_expected_matrix_counts}
  \begin{columns}[c, onlytextwidth]
  \begin{column}{0.5\textwidth}
  Removed ILC overlay particles.
  \includegraphics[height=0.8\textheight]
      {plot_factory/overlay_free_probability_matrix}
  \end{column}
  \begin{column}{0.5\textwidth}
  With overlay, categories not adapted yet.
  \includegraphics[height=0.8\textheight]
      {plot_factory/default_probability_matrix}
  \end{column}
  \end{columns}
  \end{frame}

\begin{frame}{Effect of overlay particles on the fit}
  \label{overlay_effect_on_fit}
  All fits on Higgsstrahlung events without additional background events. \\
  Category definitions are \texttt{not} adapted to the scenario with overlay events.
  \begin{columns}[c, onlytextwidth]
  \begin{column}{0.6\textwidth}
  \includegraphics[width=0.8\textwidth, keepaspectratio]
      {plot_factory/br_relative_error}
  \end{column}
  \begin{column}{0.4\textwidth}
  \includegraphics[height=0.85\textheight]
      {plot_factory/br_estimates}
  \end{column}
  \end{columns}
  \end{frame}

\begin{frame}{Optimization - Likelihood function}
  \label{many_likelihoods_optimizations}
  \begin{columns}[c, onlytextwidth]
  \begin{column}{0.6\textwidth}
  Effect of the likelihood fct definition on the uncertainties.
  The highest BR, $H \to b\bar{b}$, benefits most
  from the multinomial description.

  \includegraphics[width=0.8\textwidth, keepaspectratio]
      {plot_factory/many_br_relative_error}

  \end{column}
  \begin{column}{0.4\textwidth}
  \includegraphics[height=0.99\textheight, width=0.95\textwidth, keepaspectratio]
      {plot_factory/many_br_estimates}
  \end{column}
  \end{columns}
  \end{frame}

\begin{frame}{A fit with categories that are too correlated}
  With a highly correlated category definition/fit, the uncertainties of the \texttt{MINUIT} fit \\
  are huge, and details in the likelihood function definition matter.
  \begin{columns}[c, onlytextwidth]
  \begin{column}{0.35\textwidth}
  \includegraphics[height=0.85\textheight]
      {plot_factory/highly_correlated_probability_matrix}
  \end{column}
  \begin{column}{0.28\textwidth}
  \includegraphics[width=\textwidth]
      {plot_factory/highly_correlated}
  \end{column}
  \begin{column}{0.35\textwidth}
  \includegraphics[height=0.85\textheight]
      {plot_factory/highly_correlated_many_br_estimates}
  \end{column}
  \end{columns}
  \end{frame}

  \begin{frame}{Bias from MC sample size}
    \label{backup_bias_mc_limited}
    {%\small
      Originally, we only used 50k events per branching ratio
      for each \texttt{MC1}, \texttt{MC2}.
      The bias from limited MC samples is visible
      (SM~BR~$\neq$~EECF~minimum).
      Larger MC samples help (next slide).
    }
    \begin{columns}[c, onlytextwidth]
    \begin{column}{0.5\textwidth}
    \includegraphics[height=0.7\textheight, keepaspectratio]
        {plot_factory/toys_multinomial_original_stats/H_bb}
    \end{column}
    \begin{column}{0.5\textwidth}
      \begin{table}
        \caption{Results of a \texttt{MINUIT} fit
          on the expected event counts. In \%.}
        {%\resizebox{\textwidth}{!}{
        \InputTex{plot_factory/bias_table_original_stats.tex}
      }\end{table}
    \end{column}
    \end{columns}
    \end{frame}

\begin{frame}{Bias from MC sample size}
  {%\small
    Here, \texttt{MC1}, \texttt{MC2} are each based on
    200k events per branching ratio,
    and another 1M simulated events with SM Higgs BRs.
    The bias is almost gone.
  }
  \begin{columns}[c, onlytextwidth]
  \begin{column}{0.5\textwidth}
  \includegraphics[height=0.7\textheight, keepaspectratio]
      {plot_factory/toys_multinomial/H_bb}
  \end{column}
  \begin{column}{0.5\textwidth}
    \begin{table}
      \caption{Results of a \texttt{MINUIT} fit
        on the expected event counts. In \%.}
      {%\resizebox{\textwidth}{!}{
      \InputTex{plot_factory/bias_table.tex}
    }\end{table}
  \end{column}
  \end{columns}
  \end{frame}

\begin{frame}{Comparison with global coupling fits}
  \label{comparison_with_global}
  \begin{columns}[c, onlytextwidth]
  \begin{column}{0.55\textwidth}
  \begin{itemize}
      \item {[1], [2]} use existing analyses and combine them
          to extract a combined sensitivity for the Higgs boson couplings.
      \item {[1]} scaled to the H-20 ILC250 scenario.
      \item \textit{Multinomial} is our approach.
      \begin{itemize}
          \item A single analysis
                directly fitting the branching ratios to data.
          \item Expressed as couplings for comparison.
          \item Remember: So far signal-only.
      \end{itemize}
      \item[$\rightarrow$] Not a legitimate comparison.
          But gives hope that the results stay competitive
          even without the simplifications.
  \end{itemize}
  \end{column}
  \begin{column}{0.45\textwidth}
    \includegraphics[width=\textwidth, keepaspectratio]
        {plot_factory/comparison_with_others}
  \end{column}
  \end{columns}
  \vspace{\baselineskip}
  [1] J Tian, K Fujii~\href{https://www.sciencedirect.com/science/article/pii/S2405601415006161}
      {\color{llblue} \textit{Measurement of Higgs boson couplings at the International Linear Collider}}.
  [2] SFitter~\href{https://inspirehep.net/literature/1209590}
      {\color{llblue} \textit{Measuring Higgs Couplings at a Linear Collider}}.
  \end{frame}

\begin{frame}{Category definitions}
    \inputminted[fontsize=\scriptsize,tabsize=2,breaklines]{python}{img/extras/categories.py}
\end{frame}

% \xsection{myblue}{nJets}

\begin{frame}{Building number of jets label}%\todo[inline]{Fabricio}
    \textit{Not currently in use, but we investigate if it could help.} \\
    Trained a BDT for labeling events by the number of jets using the $y$ and $m$ variables.
    \begin{itemize}
        \item We use 20 input variables:
        \begin{itemize}
            \item The masses of individual and combined (up to 4) jets, $m$.
            \item The $y_{ij}$ values from the jet clustering algorithm.
        \end{itemize}
        \item The target used for training is the true number of hadronic jets.
        \item Trained the BDT using 100k events from the signal sample.
        \item Performed a randomized search on the following hyper parameters (\textbf{best}):
        \begin{itemize}
            \item \texttt{eta}: \quad\quad\quad\quad\quad\quad\quad [0.05, 0.10, \textbf{0.20}, 0.30, 0.35] ,
            \item \texttt{max\_depth}: \quad\quad\quad\quad [2, 3, 4, \textbf{5}, 6],
            \item \texttt{gamma}: \quad\quad\quad\quad\quad\quad [3, 1, \textbf{0.5}, 1e-01, 1e-02],
            \item \texttt{colsample\_bytree}: \hspace{0.3em} [0.5, \textbf{0.6}, 0.7].
        \end{itemize}
        % \item H$\to$ cc performance?
    \end{itemize}
\end{frame}

\begin{frame}{Building number of jets label}

     \begin{columns}[c,onlytextwidth]
      \begin{column}{0.60\textwidth}
        \includegraphics[width=0.95\textwidth]{fabricio/simple_event_vector__100k_roc.png}
      \end{column}
      \begin{column}{0.40\textwidth}
        \begin{itemize}
            \item Multiclass ROC using one-versus-rest strategy.
            \item Studies in progress to understand how well this classifier performs, particularly when tested in subsets of data (i.e. categories).
            \item Having a \# jets predicted can help us build optimal categories for Higgs BR measurements.
            % \item Comment on H cc performance
            % \item (Discuss) Include n-isolated leptons on target?
        \end{itemize}
      \end{column}
  \end{columns}

\end{frame} % Fabricio

\setcounter{framenumber}{\value{finalframe}}